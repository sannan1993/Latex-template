% Example from below
% https://www.overleaf.com/learn/latex/Learn_LaTeX_in_30_minutes
\documentclass[12pt, A4]{article} %  defines the overall class (type) of document.
%if we use book class instead of article, we can use \chapter{Let's begin} before  \section{•}
\usepackage{graphicx} %LaTeX package to import graphics
\usepackage{url} % used in Bibliology section
\graphicspath{{images/}} %configuring the graphicx package


% ///////Below is components for title page ///////////
\title{A Beginner's Guide to \LaTeX}
\author{Sannan Ahmed\thanks{Checked by Daynial Khan.}}
\date{\today} %\date{August 2022}
%/////////////////////////////


% any thing above \begin is called preamble

\begin{document}

%if you press two "enter" key then new paragraph will created

% ///////this is to make a title page///////////
\begin{titlepage}
\maketitle
\thispagestyle{empty} %this will clear the number from the title page
\end{titlepage}
%////////////////////////////////

%to add a table of contents (table of contents need to be updated (cut then Quick Build & paste then Quick Build) if any content changes.

\tableofcontents
%to add new line
\newpage

\section{This is First heading}

%normal text
Don't forget to include examples of topicalization.
They look like this:
git has been implemented and tested.

%//////////bold,italic,emphsize and underline/////////////
Some of the \textbf{greatest}  
discoveries in \underline{science} 
were made by \textbf{\textit{accident}}
not done by \emph{Experts}
%/////////////////////////////////////////////////
\subsection{How to handle topicalization}
This is how a image is inseted in latex.
% for more details on image use below link
%https://en.wikibooks.org/wiki/LaTeX/Floats,_Figures_and_Captions#:~:text=It%20is%20always%20good%20practice,command%20within%20the%20float%20environment.

% The \includegraphcs command is provided (implemented) by the graphicx package
\begin{figure}[!htb] %[!htb] is used to place image where it is in editor
	\centering
	\includegraphics[width=7cm]{avatar} 
	\caption{This is test image.}
\end{figure}

\subsection{This is second Subheading}

Mood changes when there is a topic, as well as when
there is WH-movement.  \emph{Irrealis} is the mood when
there is a non-subject topic or WH-phrase in Comp.
\emph{Realis} is the mood when there is a subject topic
or WH-phrase.

%this is how to go on new page
\newpage


\section{This is Second Heading}

% this is how bullets are inserted
\begin{itemize}
  \item The individual entries are indicated with a black dot, a so-called bullet.
  \item The text in the entries may be of any length. \\ %this is how new line is inserted
\end{itemize}

%this is how list can be inserted 
\begin{enumerate}
  \item This is the first entry in our list.
  \item The list numbers increase with each entry we add. \\
\end{enumerate}

This is how square root is inserted $E=mc^2$ % under $ is formula or use \begin{math E=mc^2 \end{math}

\subsection{This subsection describe use of table}

\begin{tabular}{|c| c| c|}
 \hline
 testing & hello from the other side & cell3 \\ 
 \hline 
 cell4 & cell5 & cell6\\  
 \hline 
 cell7 & cell8 & cell9 \\
 \hline    
\end{tabular} \\line change

paragraph change because we press two enter

\section{Bibliology and references}

We need to site the book\cite{latex2e} so that it shows in the reference, otherwise it will not show. All the reference information is in the 'refs.bib' file. We can site different types of references.

Here we site a book\cite{texbook},  article\cite{knuth:1984}, inprocessing\cite{lesk:1977}, manual\cite{evita:user} and url\cite{wiki:minV}.
% referneces
\bibliographystyle{plainurl} % We choose the "plain" reference style
\bibliography{refs} % Entries are in the refs.bib file
\end{document}